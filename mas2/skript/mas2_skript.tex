\documentclass[]{article}
\usepackage[a4paper, margin=2.5cm]{geometry}
\usepackage{amsmath}
\usepackage{amsfonts}
\usepackage{amssymb}
\usepackage{mathtools}
\usepackage{amsthm}
\usepackage[many]{tcolorbox}

\newtheorem{theorem}{Satz}
\newtheorem{lemma}{Lemma}
\newtheorem*{corollary}{Folgerung}
\newtheorem{definition}{Definition}
\newtheorem*{definition*}{Def}
\newtheorem*{remark}{Beberkung}
\newtheorem*{example}{Beispiel}

\tcolorboxenvironment{theorem}{
	colback=blue!5!white,
	boxrule=0pt,
	boxsep=1pt,
	left=2pt,right=2pt,top=2pt,bottom=2pt,
	oversize=2pt,
	sharp corners,
	before skip=\topsep,
	after skip=\topsep,
}

\tcolorboxenvironment{lemma}{
	colback=yellow!5!white,
	boxrule=0pt,
	boxsep=1pt,
	left=2pt,right=2pt,top=2pt,bottom=2pt,
	oversize=2pt,
	sharp corners,
	before skip=\topsep,
	after skip=\topsep,
}

\tcolorboxenvironment{definition}{
	colback=green!5!white,
	boxrule=0pt,
	boxsep=1pt,
	left=2pt,right=2pt,top=2pt,bottom=2pt,
	oversize=2pt,
	sharp corners,
	before skip=\topsep,
	after skip=\topsep,
}

\tcolorboxenvironment{definition*}{
	colback=green!5!white,
	boxrule=0pt,
	boxsep=1pt,
	left=2pt,right=2pt,top=2pt,bottom=2pt,
	oversize=2pt,
	sharp corners,
	before skip=\topsep,
	after skip=\topsep,
}

%opening
\title{Maß- und Wahrscheinlichkeitstheorie Skript Felsenstein W2022}
\author{Ida Hönigmann}

\begin{document}

\maketitle

\section{Produkträume und Maße}
Auf dem kartesischen Produkt von Grundmengen
\begin{align*}
	\Omega = \bigtimes_{i=1}^{n}\Omega_i
\end{align*}
wird eine Produkt-(Sigma)Algebra konstruiert, wobei $(\Omega_i, \mathcal{A}_i)$ Messräume sind. Es soll $(\Omega_i, \mathcal{A}_i)$ in $(\Omega, \mathcal{A})$ eingebettet werden.

\begin{definition*}[Projektion]
	\begin{align*}
		\pi_i : \Omega \mapsto \Omega_i \text{ mit } \pi_i(\omega_1,...,\omega_n)=\omega_i
	\end{align*}
\end{definition*}
Die Produktalgebra wird von den Projektionen erzeugt.

\begin{definition}[Produktalgebra]
	\begin{align*}
		\mathcal{A} := \bigotimes_{i=1}^{n}\mathcal{A}_i := \sigma(\pi_1,...,\pi_n)\\
		\text{d.h. } \pi_1^{-1}(A_1) \cap \pi_2^{-1}(A_2) \cap ... \cap \pi_n^{-1}(A_n) \text{ für } A_i \in \mathcal{A}_i
	\end{align*}
	erzeugen diese Algebra (bzw. $A_1\times A_2\times ...\times A_n$).
\end{definition}

Wenn $\mathcal{A}_i = \sigma(\mathcal{E}_i)$ erzeugt wird von $\mathcal{E}_i$, wird $\bigotimes\mathcal{A}_i$ von $\bigtimes\mathcal{E}_i$ erzeugt.

\begin{theorem}[1.PR]
	$\mathcal{A}_i=\sigma(\mathcal{E}_i)$ mit $\mathcal{E}_i\subset 2^{\Omega_i}$. $\Omega_i$ sei aus $\mathcal{E}_i$ monoton erreichbar ($E_{i,k} \nearrow \Omega_i$)\footnote{Man kann fordern, dass $\Omega_i \in \mathcal{E}_i$, dann ist es erfüllt.}.
	\begin{align*}
		\mathcal{E} = \{\bigtimes_{i=1}^{n}E_i, E_i\in\mathcal{E}_i\}
	\end{align*}
	Dann gilt
	\begin{align*}
		\bigotimes_{i=1}^{n}\mathcal{A}_i = \sigma(\mathcal{E}).
	\end{align*}
\end{theorem}

\begin{proof}[Beweis Satz 1.PR]
$\sigma(\mathcal{E})$ ist die kleinste Sigma-Algebra, sodass die Projektionen messbar sind: $\tilde{\mathcal{A}}$ sei Sigma-Algebra auf $\Omega = \bigtimes_i \Omega_i$.

$\pi_i$ ist $\tilde{\mathcal{A}}-\mathcal{A}_i$ messbar $\iff \sigma(\mathcal{E}) \subseteq \tilde{\mathcal{A}}$.

$\implies$: Alle $\pi_i$ seien $\tilde{\mathcal{A}}-\mathcal{A}_i$ messbar, d.h. $\pi_i^{-1}(\mathcal{E}_i) \subseteq \tilde{\mathcal{A}}$, wobei für $E_i \in \mathcal{E}_i$
\begin{align*}
	\pi_i^{-1}(E_i) = \Omega_1\times\Omega_2\times ...\times E_i\times\Omega_{i+1}\times ...\times\Omega_n
\end{align*}
Da
\begin{align*}
	\underbrace{\bigtimes_{i=1}^{n}E_i}_{\in \mathcal{E}} = \bigcap_{i=1}^{n}\pi_i^{-1}(E_i) \subseteq \tilde{\mathcal{A}} \implies \sigma(\mathcal{E}) \subseteq \tilde{\mathcal{A}}.
\end{align*}

$\impliedby$: Es gelte $\sigma(\mathcal{E}) \subseteq \tilde{\mathcal{A}}$.
Jedes $\Omega_i$ ist aus $\mathcal{E}_i$ monoton erreichbar mit $E_{i,k}\nearrow \Omega_i, k\to\infty$.
\begin{align*}
	F_k = E_{1,k}\times E_{2,k}\times ... \times E_i \times ... \times E_{n,k} \nearrow \pi_i^{-1}(E_i) \\
	\lim F_k = \bigcup_{k=1}^{\infty}F_k = \pi_i^{-1}(E_i) \in \sigma(\mathcal{E}) \subset \tilde{\mathcal{A}}
\end{align*}
Urbild vom Erzeuger in $\tilde{\mathcal{A}}$, $\pi_i^{-1}(\mathcal{E}_i)\subseteq \tilde{\mathcal{A}} \forall i$, alle $\pi_i$ sind $\tilde{\mathcal{A}}-\mathcal{A}_i$ messbar.

Alle Projektionen $\sigma(\mathcal{E})$ messbar, also $\sigma(\pi_1,...,\pi_n) \subset \sigma(\mathcal{E})$. Nach obigem setze $\tilde{\mathcal{A}}=\sigma(\pi_1,...,\pi_n) \implies \sigma(\mathcal{E}) \subseteq \sigma(\pi_1,...,\pi_n)$ also $\sigma(\mathcal{E}) = \sigma(\pi_1,...,\pi_n)$.
\end{proof}

\begin{remark}
	$\{A_1\times ...\times A_n | A_i \in \mathcal{A}_i\}$ ist keine Sigma-Algebra, da nicht Vereinigungs-stabil.
\end{remark}

Die komponentenweise Behandlung im Produktraum ist anwendbar auf n-dim. Funktionen.

\begin{corollary}
	\begin{align*}
		f_i:\Omega_0 &\mapsto \Omega_i, f=(f_1,...,f_n) =: \otimes_i f_i\\
		f: \Omega_0 &\mapsto \bigtimes_i\Omega_i = \Omega
	\end{align*}
	Dann gilt:
	\begin{align*}
		f \text{ ist } (\Omega_0,\mathcal{A}_0) \mapsto \underbrace{\bigotimes_{i=1}^{n}(\Omega_i, \mathcal{A}_i)}_{\text{Produktraum}}  \text{ messbar } \iff f_i \mathcal{A}_0-\mathcal{A}_i \text{ messbar}
	\end{align*}
\end{corollary}
\begin{proof}[Beweis Folgerung]
	$\implies$: Da $f_i=\pi_i\circ f$ und $\pi_i \otimes_j \mathcal{A}_j-\mathcal{A}_i$ messbar ist $f_i$ als Verkettung messbar.
	
	$\impliedby$: $A \in \otimes\mathcal{A}_i$ Menge aus der Produktalgebra mit $A = \times_{i=1}^{n}A_i \leftarrow$ erzeugen $\otimes_{i=1}^{n} \mathcal{A}_i$.
	\begin{align*}
		f^{-1}(A) = \bigcap_{i=1}^{n}f_i^{-1}(A_i) \in \mathcal{A}_0
	\end{align*}
	Diese Rechtecke erzeugen $\sigma(f) = \sigma(f_1,...,f_n) \subseteq \mathcal{A}_0$ also ist $f \mathcal{A}_0-\otimes\mathcal{A}_i$ messbar.
\end{proof}

Anwendung auf Borel-Algebra: $\mathcal{B}^k=\otimes_{i=1}^k\mathcal{B}$ erzeugt von den Rechtecken $\times_i (a_i,b_i]$. Die Lebesgue-Mengen werden nicht von den Produkten erzeugt: $\otimes_{i=1}^k\mathcal{L} \subsetneq \mathcal{L}_k$.

Nicht alle Nullmengen von $\mathcal{B}^k$ sind durch die Produkte mit allen Nullmengen erzeugbar.

\begin{definition*}[Schnitt]
	Besondere Mengen (für Integralberechnungen) sind die Schnitte:
	\begin{align*}
		A \in \mathcal{A} = \mathcal{A}_1 \otimes \mathcal{A}_2\\
		A_{x_1} := \{x_2 \in \Omega_2| (x_1,x_2) \in A\}\\
		A_{x_2} := \{x_1 \in \Omega_1| (x_1,x_2) \in A\}
	\end{align*}
\end{definition*}

Die Schnitte sind messbar.
\begin{theorem}[2.PR]
	$A \in \mathcal{A}_1\otimes\mathcal{A}_2$ messbar bez. Produktalgebra, dann ist $A_{x_1}\in\mathcal{A}_2$ und $A_{x_2}\in\mathcal{A}_1$ für alle $x_1$ bzw. $x_2$.
\end{theorem}

\begin{proof}
	$x_1$ sei fest. Betrachte $\mathcal{M} = \{A\subseteq \Omega_1\times\Omega_2|A_{x_1}\in\mathcal{A}_2\}$. $\mathcal{M}$ ist Sigma-Algebra: $\Omega \in \mathcal{M}$, $(A_{x_1})^c = (A^c)_{x_1}$ und $\cup A_{x_i,1}=(\cup A_i)_{x_1}$.
	
	Für die Erzeuger Mengen $A_1\times A_2 \in \mathcal{E}$ mit $A_i \in \mathcal{A}_i$
	\begin{align*}
		(A_1\times A_2)_{x_1} = \begin{cases}
									A_2, & x_1 \in A_1\\
									\emptyset, & x_1 \notin A_1
								\end{cases}
	\end{align*}
	also $(A_1\times A_2)_{x_1} \in \mathcal{A}_2$, $\mathcal{E} \subseteq \mathcal{M}$ und $\sigma(\mathcal{E}) = \mathcal{A}_1 \otimes \mathcal{A}_2 \subseteq \mathcal{M}$, alle Mengen, alle $x_1$ erfüllen die Messbarkeit-Bedingungen.
\end{proof}

Die Abbildungen $x_1 \mapsto \mu_2(A_{x_1})$ sind messbare Abbildungen.

\begin{theorem}[3.PR]
	$\mathcal{A}=\mathcal{A}_1\otimes\mathcal{A}_2$, auf $\mathcal{A}_2$ sei $\mu_2$ ein sigma-endliches Maß auf $(\Omega_2,\mathcal{A}_2,\mu_2)$. Dann ist $x_1 \mapsto \mu_2(A_{x_1})$ eine $\mathcal{A}_1$ messbare Abbildung (entsprechendes gilt auch für $x_2 \mapsto \mu_1(A_{x_2})$).
\end{theorem}
\begin{proof}
	TODO
\end{proof}

Mit diesen Funktionen wird das Produktmaß erklärt.
\begin{definition*}[Produktmaß]
	\begin{align*}
		\mu\left(\bigtimes_{i=1}^k A_i\right) = \prod_{i=1}^{k}\mu_i(A_i)
	\end{align*}
	Wenn $\Omega = \Omega_1^k$ mit $\mu_i = \mu$ gilt
	\begin{align*}
		\mu\left(\bigtimes_{i=1}^k A_i\right) = \prod_{i=1}^{k}\mu(A_i).
	\end{align*}
\end{definition*}

Dieses Maß ist eindeutig definiert, wenn $\mu_1$, $\mu_2$ sigma-endlich sind, dann sind die Funktionen $f_B(x_1)$ bzw. $f_B(x_2)$ die ''Dichten'' bezüglich den Randmaßen $\mu_1$ bzw. $\mu_2$. $\mu_1$, $\mu_2$ werden auch als marginale Maße bezeichnet.

\begin{theorem}[4.PR]
	\begin{enumerate}
		\item $\mu_2$ sei sigma-endlich. Dann definiert
		\begin{align*}
			\mu(A) = \int_{\Omega_1}\mu_2(A_{x_1})d\mu_1(x_1)
		\end{align*}
		das Produktmaß auf $(\Omega, \mathcal{A})$.
		
		\item Sind beide Maße $\mu_1$,$\mu_2$ sigma-endlich, dann ist $\mu$ eindeutig und
		\begin{align*}
			\mu(A) = \int_{\Omega_1}\mu_2(A_{x_1})d\mu_1(x_1) = \int_{\Omega_2}\mu_1(A_{x_2})d\mu_2(x_2)
		\end{align*}
	\end{enumerate}
\end{theorem}
\begin{proof}
	TODO
\end{proof}

\begin{example}
	Insbesonders bei endlichem Maß anwendbar. $X$,$Y$ stochastische Größen, dann wird eindeutig eine zweidim. Verteilung auf $\mathbb{R}^2$ durch $P[(X,Y)\in A\times B] := P[X\in A] P[Y \in B]$ Produktverteilung.
\end{example}

Verallgemeinerung der Maße von Schnitten ist die Schnittfunktion.
\begin{definition}[Schnittfunktion]
	\begin{align*}
		f:\Omega_1\times\Omega_2 \mapsto\Omega' \text{ Dann heißt }\\
		x_2 \mapsto f_{x_1}(x_2) = f(x_1,x_2) \text{ $x_1$-Schnitt}\\
		x_1 \mapsto f_{x_2}(x_1) = f(x_1,x_2) \text{ $x_2$-Schnitt}\\
	\end{align*}
	von f. (messbare Funktion $\Omega_1\times\Omega_2 \to (\Omega', \mathcal{A}')$).
\end{definition}

\begin{theorem}[5.PR]
	$f$ sei messbar $(\Omega, \mathcal{A}) \mapsto (\Omega', \mathcal{A}')$.
	Die Schnittfunktionen sind messbar:
	\begin{align*}
		f_{x_1} \text{ ist messbar } (\Omega_2,\mathcal{A}_2) \mapsto (\Omega', \mathcal{A}')\\
		f_{x_2} \text{ ist messbar } (\Omega_1,\mathcal{A}_1) \mapsto (\Omega', \mathcal{A}')
	\end{align*}
\end{theorem}
\begin{proof}
	TODO
\end{proof}

Mit den Funktionenschnitten lässt sich auch ein mehrdim. Integral "zerteilen".

\begin{theorem}[6.PR Satz von Fubini (-Tonelli)]
	Produktraum $(\Omega,\mathcal{A}, \mu) = \otimes_i (\Omega_i, \mathcal{A}_i, \mu_i)$, $f$ messbar $\Omega \mapsto \mathbb{R}$, $\mu_i$ sigma-endlich.
	Das zweidimensionale Integral von $f$ ist aufspaltbar
	\begin{align*}
		\int f d\mu = \int_{\Omega_1}\left(\int_{\Omega_2}f_{x_1}(x_2) d\mu_2\right)d\mu_1 = \int_{\Omega_2}\left(\int_{\Omega_1}f_{x_2}(x_1) d\mu_1\right)d\mu_2
	\end{align*}
	wenn eine der folgenden Bedingungen gilt:
	\begin{itemize}
		\item $f\geq 0$: Dann ist $x_2\mapsto \phi_2(x_2) = \int_{\Omega_1}f_{x_2}d\mu_1$ messbar $\mathcal{A}_2$ und $x_1\mapsto \phi_1(x_1) = \int_{\Omega_2}f_{x_1}d\mu_2$ messbar $\mathcal{A}_1$
		\item $f$ ist integrierbar, $\int f d\mu < \infty$: Dann sind $f_{x_1}$ $\mu_2$-integrierbar $[\mu_1]$ f.ü. und $f_{x_2}$ $\mu_1$-integrierbar $[\mu_2]$ f.ü.
		\item $\int_{\Omega_1}\int_{\Omega_2}|f_{x_1}|d\mu_2 d\mu_1 < \infty$ oder $\int_{\Omega_2}\int_{\Omega_1}|f_{x_2}|d\mu_1 d\mu_2 < \infty$: Daraus folgt $f$ integrierbar.
	\end{itemize}
\end{theorem}
\begin{proof}
	TODO
\end{proof}

Durch Iteration gilt die Fubini-Schnitt Konstruktion auch für mehrdimensionale $k\in\mathbb{N}$ Integrale:

Wenn $f:\Omega\mapsto\mathbb{R}$ messbar mit $\Omega=\Omega_1\times ...\times\Omega_n$
\begin{align*}
	\int f d\mu = \int_{\Omega_1}\left(\int_{\Omega_2\times ...\times\Omega_n} f_{x_1} d\mu_2\otimes ...\otimes \mu_k\right) d\mu_1(x_1)
\end{align*}

Viele Folgerungen: Doppelreihen-Satz $\sum_i\sum_j a_{ij} = \sum_j\sum_i a_{ij}$ Kriterien für Konvergenz

\begin{corollary}[Maße mit Dichten bezüglich dem Produktmaß]
	$\Omega = \Omega_1\times\Omega_2$, $\mathcal{A}=\mathcal{A}_1\otimes\mathcal{A}_2$,  $\mu=\mu_1\otimes\mu_2$
	
	$\nu$ sei absolut stetig bzgl. $\mu$ ($\nu \ll \mu$), es existiert ein $f=\frac{d\nu}{d\mu}\geq 0$ mit $\nu(A)=\int_A f d\mu$ ($f$ integrierbar)
	
	$\mu$ sei sigma-endlich, $\nu$ erzeugt ein $\nu_1 \ll \mu_1$ auf $(\Omega_1, \mathcal{A}_1)$ und ein $\nu_2 \ll \mu_2$ auf ($\Omega_2, \mathcal{A}_2$) mit den Dichten $\phi_1, \phi_2$.
	\begin{align*}
		A \in \mathcal{A}_1: \nu_1(A_1) = \int_{A_1}\phi_1(x_1)d\mu_1 = \int_{A_1} \int_{\Omega_2} f_{x_1}(x_2) d\mu_2 d\mu_1
	\end{align*}
\end{corollary}

\begin{example}
	2 dim SG mit Gleichverteilung auf dem Einheitskreis $P[(X,Y)\in K] = 1$. Dichte $f$ bezgl. $\lambda^2 = \lambda_1\otimes\lambda_1$. $f(x,y)=\frac{1}{\pi}1_K$.
	\begin{align*}
		\phi_1(x)=\int f_X(y)d\lambda = \int_{-\sqrt{1-x^2}}^{\sqrt{1-x^2}}\frac{1}{\pi} d\lambda(y) = \frac{2\sqrt{1-x^2}}{\pi}
	\end{align*}
	(Symmetrie: $\phi_2(y) = \frac{2\sqrt{1-y^2}}{\pi}$)
	
	Randverteilung $\nu_1$ $P[X\in A] = \int_A \frac{2}{\pi}\sqrt{1-y^2} d\lambda(y)$
	
	$\nu$ ist nicht das Produktmaß $\nu \neq \nu_1 \otimes \nu_2$ außer $f(x,y)=f_1(x)f_2(y)$.
\end{example}

\begin{definition*}[Ordinatenmenge, Graph]
	Die Punkte unter einer positiven Funktion $f\geq 0$ heißt Ordinatenmenge $O_f=\{(x,y)|0\leq <\leq f(x)\}$ und der Graph ist $\Gamma_f=\{(x,f(x))|x\in\Omega\}$.
\end{definition*}

Für sigma-endliches Maß $\mu$ und $f$ messbar, dann ist $O_f$ eine bezüglich $\mathcal{A}\otimes\mathcal{B}$ messbare Menge und $\int f d\mu = (\mu \otimes \lambda)(O_f)$. Der Graph ist eine Nullmenge, $(\mu \otimes \lambda)(\Gamma_f) =0$

\subsection{$\infty$-dim. Produkträume}
Die Konstruktion der $\infty$-dim. Produkt-Messräume ist der endl. dim. Konstruktion entsprechend. $I$ sei Index-Menge,
\begin{align*}
	\otimes_{i\in I}\mathcal{A}_i := \sigma(\pi_i, i\in I) && \pi_i ... \text{ Projektion auf } \Omega_i
\end{align*}
beispielsweise $\otimes_{i\in I}\mathcal{B}$ auf $\Omega=\mathbb{R}^I$ (Funktionenraum).

\begin{definition*}[Zylinder,Pfeiler]
	$Z \subseteq \Omega^I$ heißt Zylinder, wenn $Z=\pi_j^{-1}(C)=C\times\Omega^{j^C}$ wobei $J\subseteq I$ eine endliche Teilindexmenge ist und $C$ die endlich dim. Basis des Zylinders ist;
	und Pfeiler, wenn $C$ ein Rechteck $\times_{i \in J}A_i$ ist.
\end{definition*}

Auf den Pfeilern lässt sich das n-dim. Produktmaß erklären. $P^n(C)=\pi_{j\in J}P(A_j)$

Im abzählbar unendlichen Fall ist die Vorgangsweise ähnlich, wie im endl. dim. Fall:

\begin{theorem}[7.PR]
	$X_i$ sei eine Folge von SGn auf $(\Omega_i,\mathcal{A}_i)$ mit gegebener gemeinsamer Verteilung
	\begin{align*}
		P_i(A) = P[X_j\in A_j, j\leq i, X_i \in A]
	\end{align*}
	für $A_j\in\mathcal{A}_j$ und $A\in\mathcal{A}_i$. D.h. die gemeinsame Verteilung von $(X_1,...,X_n)$ sind auf $(\times_i\Omega_i, \otimes_i\mathcal{A}_i)$ mit
	\begin{align*}
		P[(X_1,...,X_n)\in A_n, X_i \in \mathbb{R}, i>n] = P[(X_1,...,X_n) \in A_n]
	\end{align*}
	also für Zylinder $Z = \pi_{1,...,n}^{-1}(C_n)$, $C_n \in \otimes_i\mathcal{A}_i$ gilt $P[Z]=P_n(Cn)$.
	
	Wenn diese Wahrscheinlichkeitsverteilung verträglich sind, d.h. die Randverteilungen (bzw. alle Teilmengen endl.) eindeutig sind, lässt sich obiges Prinzip verallgemeinern.
\end{theorem}

\begin{theorem}[8.PR Satz von Kolmogoroff]
	content...
\end{theorem}

\end{document}
