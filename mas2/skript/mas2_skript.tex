\documentclass[]{article}
\usepackage[a4paper, margin=2.5cm]{geometry}
\usepackage{amsmath}
\usepackage{amsfonts}
\usepackage{amssymb}
\usepackage{mathtools}
\usepackage{amsthm}
\usepackage[many]{tcolorbox}

\newtheorem{theorem}{Satz}
\newtheorem{lemma}{Lemma}
\newtheorem*{corollary}{Folgerung}
\newtheorem{definition}{Definition}
\newtheorem*{definition*}{Def}
\newtheorem*{remark}{Beberkung}
\newtheorem*{example}{Beispiel}

\tcolorboxenvironment{theorem}{
	colback=blue!5!white,
	boxrule=0pt,
	boxsep=1pt,
	left=2pt,right=2pt,top=2pt,bottom=2pt,
	oversize=2pt,
	sharp corners,
	before skip=\topsep,
	after skip=\topsep,
}

\tcolorboxenvironment{lemma}{
	colback=yellow!5!white,
	boxrule=0pt,
	boxsep=1pt,
	left=2pt,right=2pt,top=2pt,bottom=2pt,
	oversize=2pt,
	sharp corners,
	before skip=\topsep,
	after skip=\topsep,
}

\tcolorboxenvironment{definition}{
	colback=green!5!white,
	boxrule=0pt,
	boxsep=1pt,
	left=2pt,right=2pt,top=2pt,bottom=2pt,
	oversize=2pt,
	sharp corners,
	before skip=\topsep,
	after skip=\topsep,
}

\tcolorboxenvironment{definition*}{
	colback=green!5!white,
	boxrule=0pt,
	boxsep=1pt,
	left=2pt,right=2pt,top=2pt,bottom=2pt,
	oversize=2pt,
	sharp corners,
	before skip=\topsep,
	after skip=\topsep,
}

%opening
\title{Maß- und Wahrscheinlichkeitstheorie Skript Felsenstein W2022}
\author{Ida Hönigmann}

\begin{document}

\maketitle

\section{Produkträume und Maße}
Auf dem kartesischen Produkt von Grundmengen
\begin{align*}
	\Omega = \bigtimes_{i=1}^{n}\Omega_i
\end{align*}
wird eine Produkt-(Sigma)Algebra konstruiert, wobei $(\Omega_i, \mathcal{A}_i)$ Messräume sind. Es soll $(\Omega_i, \mathcal{A}_i)$ in $(\Omega, \mathcal{A})$ eingebettet werden.

\begin{definition*}[Projektion]
	\begin{align*}
		\pi_i : \Omega \mapsto \Omega_i \text{ mit } \pi_i(\omega_1,...,\omega_n)=\omega_i
	\end{align*}
\end{definition*}
Die Produktalgebra wird von den Projektionen erzeugt.

\begin{definition}[Produktalgebra]
	\begin{align*}
		\mathcal{A} := \bigotimes_{i=1}^{n}\mathcal{A}_i := \sigma(\pi_1,...,\pi_n)\\
		\text{d.h. } \pi_1^{-1}(A_1) \cap \pi_2^{-1}(A_2) \cap ... \cap \pi_n^{-1}(A_n) \text{ für } A_i \in \mathcal{A}_i
	\end{align*}
	erzeugen diese Algebra (bzw. $A_1\times A_2\times ...\times A_n$).
\end{definition}

Wenn $\mathcal{A}_i = \sigma(\mathcal{E}_i)$ erzeugt wird von $\mathcal{E}_i$, wird $\bigotimes\mathcal{A}_i$ von $\bigtimes\mathcal{E}_i$ erzeugt.

\begin{theorem}[1.PR]
	$\mathcal{A}_i=\sigma(\mathcal{E}_i)$ mit $\mathcal{E}_i\subset 2^{\Omega_i}$. $\Omega_i$ sei aus $\mathcal{E}_i$ monoton erreichbar ($E_{i,k} \nearrow \Omega_i$)\footnote{Man kann fordern, dass $\Omega_i \in \mathcal{E}_i$, dann ist es erfüllt.}.
	\begin{align*}
		\mathcal{E} = \{\bigtimes_{i=1}^{n}E_i, E_i\in\mathcal{E}_i\}
	\end{align*}
	Dann gilt
	\begin{align*}
		\bigotimes_{i=1}^{n}\mathcal{A}_i = \sigma(\mathcal{E}).
	\end{align*}
\end{theorem}

\begin{proof}[Beweis Satz 1.PR]
$\sigma(\mathcal{E})$ ist die kleinste Sigma-Algebra, sodass die Projektionen messbar sind: $\tilde{\mathcal{A}}$ sei Sigma-Algebra auf $\Omega = \bigtimes_i \Omega_i$.

$\pi_i$ ist $\tilde{\mathcal{A}}-\mathcal{A}_i$ messbar $\iff \sigma(\mathcal{E}) \subseteq \tilde{\mathcal{A}}$.

$\implies$: Alle $\pi_i$ seien $\tilde{\mathcal{A}}-\mathcal{A}_i$ messbar, d.h. $\pi_i^{-1}(\mathcal{E}_i) \subseteq \tilde{\mathcal{A}}$, wobei für $E_i \in \mathcal{E}_i$
\begin{align*}
	\pi_i^{-1}(E_i) = \Omega_1\times\Omega_2\times ...\times E_i\times\Omega_{i+1}\times ...\times\Omega_n
\end{align*}
Da
\begin{align*}
	\underbrace{\bigtimes_{i=1}^{n}E_i}_{\in \mathcal{E}} = \bigcap_{i=1}^{n}\pi_i^{-1}(E_i) \subseteq \tilde{\mathcal{A}} \implies \sigma(\mathcal{E}) \subseteq \tilde{\mathcal{A}}.
\end{align*}

$\impliedby$: Es gelte $\sigma(\mathcal{E}) \subseteq \tilde{\mathcal{A}}$.
Jedes $\Omega_i$ ist aus $\mathcal{E}_i$ monoton erreichbar mit $E_{i,k}\nearrow \Omega_i, k\to\infty$.
\begin{align*}
	F_k = E_{1,k}\times E_{2,k}\times ... \times E_i \times ... \times E_{n,k} \nearrow \pi_i^{-1}(E_i) \\
	\lim F_k = \bigcup_{k=1}^{\infty}F_k = \pi_i^{-1}(E_i) \in \sigma(\mathcal{E}) \subset \tilde{\mathcal{A}}
\end{align*}
Urbild vom Erzeuger in $\tilde{\mathcal{A}}$, $\pi_i^{-1}(\mathcal{E}_i)\subseteq \tilde{\mathcal{A}} \forall i$, alle $\pi_i$ sind $\tilde{\mathcal{A}}-\mathcal{A}_i$ messbar.

Alle Projektionen $\sigma(\mathcal{E})$ messbar, also $\sigma(\pi_1,...,\pi_n) \subset \sigma(\mathcal{E})$. Nach obigem setze $\tilde{\mathcal{A}}=\sigma(\pi_1,...,\pi_n) \implies \sigma(\mathcal{E}) \subseteq \sigma(\pi_1,...,\pi_n)$ also $\sigma(\mathcal{E}) = \sigma(\pi_1,...,\pi_n)$.
\end{proof}

\begin{remark}
	$\{A_1\times ...\times A_n | A_i \in \mathcal{A}_i\}$ ist keine Sigma-Algebra, da nicht Vereinigungs-stabil.
\end{remark}

Die komponentenweise Behandlung im Produktraum ist anwendbar auf n-dim. Funktionen.

\begin{corollary}
	\begin{align*}
		f_i:\Omega_0 &\mapsto \Omega_i, f=(f_1,...,f_n) =: \otimes_i f_i\\
		f: \Omega_0 &\mapsto \bigtimes_i\Omega_i = \Omega
	\end{align*}
	Dann gilt:
	\begin{align*}
		f \text{ ist } (\Omega_0,\mathcal{A}_0) \mapsto \underbrace{\bigotimes_{i=1}^{n}(\Omega_i, \mathcal{A}_i)}_{\text{Produktraum}}  \text{ messbar } \iff f_i \mathcal{A}_0-\mathcal{A}_i \text{ messbar}
	\end{align*}
\end{corollary}
\begin{proof}[Beweis Folgerung]
	$\implies$: Da $f_i=\pi_i\circ f$ und $\pi_i \otimes_j \mathcal{A}_j-\mathcal{A}_i$ messbar ist $f_i$ als Verkettung messbar.
	
	$\impliedby$: $A \in \otimes\mathcal{A}_i$ Menge aus der Produktalgebra mit $A = \times_{i=1}^{n}A_i \leftarrow$ erzeugen $\otimes_{i=1}^{n} \mathcal{A}_i$.
	\begin{align*}
		f^{-1}(A) = \bigcap_{i=1}^{n}f_i^{-1}(A_i) \in \mathcal{A}_0
	\end{align*}
	Diese Rechtecke erzeugen $\sigma(f) = \sigma(f_1,...,f_n) \subseteq \mathcal{A}_0$ also ist $f \mathcal{A}_0-\otimes\mathcal{A}_i$ messbar.
\end{proof}

Anwendung auf Borel-Algebra: $\mathcal{B}^k=\otimes_{i=1}^k\mathcal{B}$ erzeugt von den Rechtecken $\times_i (a_i,b_i]$. Die Lebesgue-Mengen werden nicht von den Produkten erzeugt: $\otimes_{i=1}^k\mathcal{L} \subsetneq \mathcal{L}_k$.

Nicht alle Nullmengen von $\mathcal{B}^k$ sind durch die Produkte mit allen Nullmengen erzeugbar.

\begin{definition*}[Schnitt]
	Besondere Mengen (für Integralberechnungen) sind die Schnitte:
	\begin{align*}
		A \in \mathcal{A} = \mathcal{A}_1 \otimes \mathcal{A}_2\\
		A_{x_1} := \{x_2 \in \Omega_2| (x_1,x_2) \in A\}\\
		A_{x_2} := \{x_1 \in \Omega_1| (x_1,x_2) \in A\}
	\end{align*}
\end{definition*}

Die Schnitte sind messbar.
\begin{theorem}[2.PR]
	$A \in \mathcal{A}_1\otimes\mathcal{A}_2$ messbar bez. Produktalgebra, dann ist $A_{x_1}\in\mathcal{A}_2$ und $A_{x_2}\in\mathcal{A}_1$ für alle $x_1$ bzw. $x_2$.
\end{theorem}

\begin{proof}
	$x_1$ sei fest. Betrachte $\mathcal{M} = \{A\subseteq \Omega_1\times\Omega_2|A_{x_1}\in\mathcal{A}_2\}$. $\mathcal{M}$ ist Sigma-Algebra: $\Omega \in \mathcal{M}$, $(A_{x_1})^c = (A^c)_{x_1}$ und $\cup A_{x_i,1}=(\cup A_i)_{x_1}$.
	
	Für die Erzeuger Mengen $A_1\times A_2 \in \mathcal{E}$ mit $A_i \in \mathcal{A}_i$
	\begin{align*}
		(A_1\times A_2)_{x_1} = \begin{cases}
									A_2, & x_1 \in A_1\\
									\emptyset, & x_1 \notin A_1
								\end{cases}
	\end{align*}
	also $(A_1\times A_2)_{x_1} \in \mathcal{A}_2$, $\mathcal{E} \subseteq \mathcal{M}$ und $\sigma(\mathcal{E}) = \mathcal{A}_1 \otimes \mathcal{A}_2 \subseteq \mathcal{M}$, alle Mengen, alle $x_1$ erfüllen die Messbarkeit-Bedingungen.
\end{proof}

Die Abbildungen $x_1 \mapsto \mu_2(A_{x_1})$ sind messbare Abbildungen.

\begin{theorem}[3.PR]
	$\mathcal{A}=\mathcal{A}_1\otimes\mathcal{A}_2$, auf $\mathcal{A}_2$ sei $\mu_2$ ein sigma-endliches Maß auf $(\Omega_2,\mathcal{A}_2,\mu_2)$. Dann ist $x_1 \mapsto \mu_2(A_{x_1})$ eine $\mathcal{A}_1$ messbare Abbildung (entsprechendes gilt auch für $x_2 \mapsto \mu_1(A_{x_2})$).
\end{theorem}
\begin{proof}
	Für ein $A\in\mathcal{A}$ sei $f_A(x_1)=\mu_2(Ax_1)$
	\begin{enumerate}
		\item $\mu_2$ sei endlich. Betrachte $\mathcal{D}=\{E\in\mathcal{A}|f_E ist \mathcal{A}_1 \text{ Borel-messbar} (\mathbb{R}^+,\mathcal{B}^+)\}$
		
		$\mathcal{D}$ ist ein Dynkin-System:
		\begin{itemize}
			\item $\Omega \in \mathcal{D}$, da $\Omega_{x_1}=\Omega_2 \in \mathcal{A}_2 \forall x_1 : f_\Omega \equiv \mu_2(\Omega_2)$ konstant
			
			\item $A\subseteq\mathcal{B}$ und $f_A,f_B$ messbar
			
			$A_{x_1} \subseteq B_{x_1}$ und $\mu_2(B_{x_1}\setminus A_{x_1}) = \mu_2(B_{x_1}) - \mu_2(A_{x_1}) = f_B(x_1) - f_A(x_1)$ ist messbar als Differenz messbarer Funktionen.
			
			\item $A_i \in \mathcal{A}$ und disjunkt, $B=\bigcup_{i}A_i$
			
			Wenn $A_i \in \mathcal{D}$, $B_{x_1} = \bigcup A_{i,x_1}$
			\begin{align*}
				\mu_2(B_{x_1}) = f_B(x_1) = \sum_i f_{A_i}(x_1) = \mu_2(\bigcup A_{x_1})
			\end{align*}
			Eine abzählbare Summe messbarer Funktionen ist messbar (Darstellung jeder messbarer Funk $f=\sum_j c_j 1_{c_j}$).
		\end{itemize}
	
		Jede Menge aus $\mathcal{E}=\{A_1\times A_2 | A_i \in \mathcal{A_i}\}$ (dem Erzeuger von $\mathcal{A}$) ist auch in $\mathcal{D}$, weil $f_{A_1\times A_2}(x_1) = \mu_2((A_1\times A_2)_{x_1}) = \mu_2(A_2) 1_{A_1}(x_1)$. $c1$ ist messbar.
		
		$\mathcal{E}$ ist ein durchschnitt-stabiler Erzeuger und $\mathcal{E}\subseteq\mathcal{D}\subseteq\mathcal{A}=\sigma(\mathcal{E})$ also $\mathcal{A}=\mathcal{D}$, alle solchen Funktionen sind messbar bez. $\mathcal{A_1}$.
		
		\item Ist $\mu_2$ sigma-endlich, es gibt eine Folge $A_{2,i}\nearrow\Omega_2$ mit endlichem Maß, daher auch eine disjunkte Folge ($D_n$), die eine Zerlegung von $\Omega_2$ sind und $\mu_2(B) = \mu_2(\bigcup_n(D_n\cap B)) = \sum_n \mu_2(B\cap D_n)$.
		\begin{align*}
			f_B(x_1) = \mu_2(B_{x_1}) = \sum_n \underbrace{\mu_2(B_{x_1} \cap D_n)}_{<\infty} = \sum_n f_{(B_{x_1}\cap D_n)}(x_1)
		\end{align*}
		also Summe messbarer Funktionen.
	\end{enumerate}
\end{proof}

Mit diesen Funktionen wird das Produktmaß erklärt.
\begin{definition*}[Produktmaß]
	\begin{align*}
		\mu\left(\bigtimes_{i=1}^k A_i\right) = \prod_{i=1}^{k}\mu_i(A_i)
	\end{align*}
	Wenn $\Omega = \Omega_1^k$ mit $\mu_i = \mu$ gilt
	\begin{align*}
		\mu\left(\bigtimes_{i=1}^k A_i\right) = \prod_{i=1}^{k}\mu(A_i).
	\end{align*}
\end{definition*}

Dieses Maß ist eindeutig definiert, wenn $\mu_1$, $\mu_2$ sigma-endlich sind, dann sind die Funktionen $f_B(x_1)$ bzw. $f_B(x_2)$ die ''Dichten'' bezüglich den Randmaßen $\mu_1$ bzw. $\mu_2$. $\mu_1$, $\mu_2$ werden auch als marginale Maße bezeichnet.

\begin{theorem}[4.PR]
	\begin{enumerate}
		\item $\mu_2$ sei sigma-endlich. Dann definiert
		\begin{align*}
			\mu(A) = \int_{\Omega_1}\mu_2(A_{x_1})d\mu_1(x_1)
		\end{align*}
		das Produktmaß auf $(\Omega, \mathcal{A})$.
		
		\item Sind beide Maße $\mu_1$,$\mu_2$ sigma-endlich, dann ist $\mu$ eindeutig und
		\begin{align*}
			\mu(A) = \int_{\Omega_1}\mu_2(A_{x_1})d\mu_1(x_1) = \int_{\Omega_2}\mu_1(A_{x_2})d\mu_2(x_2)
		\end{align*}
	\end{enumerate}
\end{theorem}
\begin{proof}
	$f_A(x_1)=\mu_2(A_{x_1})$ ist eine $\mathcal{A}_1$-messbare Funktion $\forall A$ und durch die Additivität $f_{\cup A_i} = \sum f_{A_i}$ ist $\mu$ ein Maß auf $(\Omega, \mathcal{A})$.
	
	Für $A=A_1\times A_2$ gilt
	\begin{align*}
		\mu(A_1\times A_2) = \int_{\Omega_1} \underbrace{\mu_2((A_1\times A_2)_{x_1})}_{\mu_2(A_2) 1_{A_1}(x_1)} d\mu_1 = \mu_2(A_2) \underbrace{\int_{A_1} d\mu_1}_{\mu_1(A_1)}
	\end{align*}
	ist Produktmaß.
	
	$\mu$ ist auf dem (durchschnitts-stabilen) Erzeuger definiert, da die Maße sigma-endlich sind, ist $\mu$ eindeutig (Eindeutigkeitssatz). Vice versa gelten alle Gleichungen analog bez. $\mu_2$.
\end{proof}

\begin{example}
	Insbesonders bei endlichem Maß anwendbar. $X$,$Y$ stochastische Größen, dann wird eindeutig eine zweidim. Verteilung auf $\mathbb{R}^2$ durch $P[(X,Y)\in A\times B] := P[X\in A] P[Y \in B]$ Produktverteilung.
\end{example}

Verallgemeinerung der Maße von Schnitten ist die Schnittfunktion.
\begin{definition}[Schnittfunktion]
	\begin{align*}
		f:\Omega_1\times\Omega_2 \mapsto\Omega' \text{ Dann heißt }\\
		x_2 \mapsto f_{x_1}(x_2) = f(x_1,x_2) \text{ $x_1$-Schnitt}\\
		x_1 \mapsto f_{x_2}(x_1) = f(x_1,x_2) \text{ $x_2$-Schnitt}\\
	\end{align*}
	von f. (messbare Funktion $\Omega_1\times\Omega_2 \to (\Omega', \mathcal{A}')$).
\end{definition}

\begin{theorem}[5.PR]
	$f$ sei messbar $(\Omega, \mathcal{A}) \mapsto (\Omega', \mathcal{A}')$.
	Die Schnittfunktionen sind messbar:
	\begin{align*}
		f_{x_1} \text{ ist messbar } (\Omega_2,\mathcal{A}_2) \mapsto (\Omega', \mathcal{A}')\\
		f_{x_2} \text{ ist messbar } (\Omega_1,\mathcal{A}_1) \mapsto (\Omega', \mathcal{A}')
	\end{align*}
\end{theorem}
\begin{proof}
	$A'\in\mathcal{A}'$
	\begin{align*}
		f_{x_1}^{-1}(A') = \{x_2\in\Omega_2 | f(x_1,x_2) \in A'\} = \{x_2\in \Omega_2 | (x_1,x_2) \in f^{-1}(A')\} = (\underbrace{f^{-1}(A')}_{\text{ mb } \in \mathcal{A}})_{x_1}
	\end{align*}
	jede Schnittmenge ist messbar.
	
	Analog für $f_{x_2}^{-1}(A') \in \mathcal{A}_1$.
\end{proof}

Mit den Funktionenschnitten lässt sich auch ein mehrdim. Integral "zerteilen".

\begin{theorem}[6.PR Satz von Fubini (-Tonelli)]
	Produktraum $(\Omega,\mathcal{A}, \mu) = \otimes_i (\Omega_i, \mathcal{A}_i, \mu_i)$, $f$ messbar $\Omega \mapsto \mathbb{R}$, $\mu_i$ sigma-endlich.
	Das zweidimensionale Integral von $f$ ist aufspaltbar
	\begin{align*}
		\int f d\mu = \int_{\Omega_1}\left(\int_{\Omega_2}f_{x_1}(x_2) d\mu_2\right)d\mu_1 = \int_{\Omega_2}\left(\int_{\Omega_1}f_{x_2}(x_1) d\mu_1\right)d\mu_2
	\end{align*}
	wenn eine der folgenden Bedingungen gilt:
	\begin{enumerate}
		\item $f\geq 0$: Dann ist $x_2\mapsto \phi_2(x_2) = \int_{\Omega_1}f_{x_2}d\mu_1$ messbar $\mathcal{A}_2$ und $x_1\mapsto \phi_1(x_1) = \int_{\Omega_2}f_{x_1}d\mu_2$ messbar $\mathcal{A}_1$
		\item $f$ ist integrierbar, $\int f d\mu < \infty$: Dann sind $f_{x_1}$ $\mu_2$-integrierbar $[\mu_1]$ f.ü. und $f_{x_2}$ $\mu_1$-integrierbar $[\mu_2]$ f.ü.
		\item $\int_{\Omega_1}\int_{\Omega_2}|f_{x_1}|d\mu_2 d\mu_1 < \infty$ oder $\int_{\Omega_2}\int_{\Omega_1}|f_{x_2}|d\mu_1 d\mu_2 < \infty$: Daraus folgt $f$ integrierbar.
	\end{enumerate}
\end{theorem}
\begin{proof}
	\begin{enumerate}
		\item $f \geq 0:$ 4 Schritte des Integralaufbaus
		
		$f = 1_A, A \in \mathcal{A}$
		\begin{align*}
			x_2 \mapsto \phi_2(x_2) = \int_{\Omega_1} \underbrace{(1_A)_{x_2}}_{1_{A_{x_2}}} d\mu_1 = \mu_1(A_{x_2})
		\end{align*}
	
		$\phi_2$ ist messbar (laut Satz PR5) und nach Satz PR7 gilt
		\begin{align*}
			\int f d\mu = \mu(A) = \int_{\Omega_2} \mu_1(A_{x_2}) d\mu_2(x_2) = \int_{\Omega_2} \phi_2 d\mu_2
		\end{align*}
	
		Für einfache Funktionen $f = \sum_{i=1}^{n}\alpha_i 1_{A_i}$ ergibt sich das aus der Linearität:
		\begin{align*}
			f_{x_2} = \sum \alpha_i 1_{(A_i)_{x_2}}(x_1)
		\end{align*}
	
		Schritt 3: $f_2\nearrow f, f_n$... einfache Funktionen, dann gilt $(f_n)_{x_2}\nearrow f_{x_2}$ (wieder aus der Darstellung $f=\sum_{i=1}^{\infty}\alpha_i 1_{A_i}$ + monotone Konvergenz ablesbar)
		
		\item $f$ integrierbar, betrachte Positiv- und Negativteil
		\begin{align*}
			max(0,f)_{x_2} = max(0, f_{x_2}) = (f^+)_{x_2}\\
			\text{also } (f^+)_{x_2} = (f_{x_2})^+ \text{ und } f_{x_2}^- = (f^-)_{x_2} \text{ mit } |f|_{x_2} = |f_{x_2}|
		\end{align*}
	
		Wegen $\int |f| d\mu < \infty$ gilt wegen oben für $|f|$:
		\begin{align*}
			\int |f| d\mu = \int_{\Omega_1} \int_{\Omega_2} |f|_{x_1} d\mu_2 d\mu_1 = \int_{\Omega_1} \underbrace{\int_{\Omega_2} |f_{x_1}| d\mu_2}_{\phi_1} d\mu_1 < \infty
		\end{align*}
	
		Daher muss $\phi_1$ integrierbar sein (wie auch $\phi_2$). Integrierbarkeit erfordert auch $(f^+), (f^-)$ integrierbar sind, aufgespalten $f = f^+-f^-$ und $f_{x_i} = f_{x_i}^+ - f_{x_i}^-$. Nach 1. für $f^+, f^-$ getrennt ergibt
		\begin{align*}
			\int f d\mu = \int f^+ - \int f^- = \int_{\Omega_1} \int_{\Omega_2} f_{x_1}^+ d\mu_2 d\mu_1 - \int\int f^- d\mu_2 d\mu_1
		\end{align*}
	
		\item impliziert $\int_{\Omega_2} f_{x_1}^+ d\mu_2 < \infty, \int_{\Omega_2} f_{x_1}^- d\mu_2 < \infty$ und somit $f$ integrierbar und 2.
	\end{enumerate}
\end{proof}

Durch Iteration gilt die Fubini-Schnitt Konstruktion auch für mehrdimensionale $k\in\mathbb{N}$ Integrale:

Wenn $f:\Omega\mapsto\mathbb{R}$ messbar mit $\Omega=\Omega_1\times ...\times\Omega_n$
\begin{align*}
	\int f d\mu = \int_{\Omega_1}\left(\int_{\Omega_2\times ...\times\Omega_n} f_{x_1} d\mu_2\otimes ...\otimes \mu_k\right) d\mu_1(x_1)
\end{align*}

Viele Folgerungen: Doppelreihen-Satz $\sum_i\sum_j a_{ij} = \sum_j\sum_i a_{ij}$ Kriterien für Konvergenz

\begin{corollary}[Maße mit Dichten bezüglich dem Produktmaß]
	$\Omega = \Omega_1\times\Omega_2$, $\mathcal{A}=\mathcal{A}_1\otimes\mathcal{A}_2$,  $\mu=\mu_1\otimes\mu_2$
	
	$\nu$ sei absolut stetig bzgl. $\mu$ ($\nu \ll \mu$), es existiert ein $f=\frac{d\nu}{d\mu}\geq 0$ mit $\nu(A)=\int_A f d\mu$ ($f$ integrierbar)
	
	$\mu$ sei sigma-endlich, $\nu$ erzeugt ein $\nu_1 \ll \mu_1$ auf $(\Omega_1, \mathcal{A}_1)$ und ein $\nu_2 \ll \mu_2$ auf ($\Omega_2, \mathcal{A}_2$) mit den Dichten $\phi_1, \phi_2$.
	\begin{align*}
		A \in \mathcal{A}_1: \nu_1(A_1) = \int_{A_1}\phi_1(x_1)d\mu_1 = \int_{A_1} \int_{\Omega_2} f_{x_1}(x_2) d\mu_2 d\mu_1
	\end{align*}
\end{corollary}

\begin{example}
	2 dim SG mit Gleichverteilung auf dem Einheitskreis $P[(X,Y)\in K] = 1$. Dichte $f$ bezgl. $\lambda^2 = \lambda_1\otimes\lambda_1$. $f(x,y)=\frac{1}{\pi}1_K$.
	\begin{align*}
		\phi_1(x)=\int f_X(y)d\lambda = \int_{-\sqrt{1-x^2}}^{\sqrt{1-x^2}}\frac{1}{\pi} d\lambda(y) = \frac{2\sqrt{1-x^2}}{\pi}
	\end{align*}
	(Symmetrie: $\phi_2(y) = \frac{2\sqrt{1-y^2}}{\pi}$)
	
	Randverteilung $\nu_1$ $P[X\in A] = \int_A \frac{2}{\pi}\sqrt{1-y^2} d\lambda(y)$
	
	$\nu$ ist nicht das Produktmaß $\nu \neq \nu_1 \otimes \nu_2$ außer $f(x,y)=f_1(x)f_2(y)$.
\end{example}

\begin{definition*}[Ordinatenmenge, Graph]
	Die Punkte unter einer positiven Funktion $f\geq 0$ heißt Ordinatenmenge $O_f=\{(x,y)|0\leq <\leq f(x)\}$ und der Graph ist $\Gamma_f=\{(x,f(x))|x\in\Omega\}$.
\end{definition*}

Für sigma-endliches Maß $\mu$ und $f$ messbar, dann ist $O_f$ eine bezüglich $\mathcal{A}\otimes\mathcal{B}$ messbare Menge und $\int f d\mu = (\mu \otimes \lambda)(O_f)$. Der Graph ist eine Nullmenge, $(\mu \otimes \lambda)(\Gamma_f) =0$

\subsection{$\infty$-dim. Produkträume}
Die Konstruktion der $\infty$-dim. Produkt-Messräume ist der endl. dim. Konstruktion entsprechend. $I$ sei Index-Menge,
\begin{align*}
	\otimes_{i\in I}\mathcal{A}_i := \sigma(\pi_i, i\in I) && \pi_i ... \text{ Projektion auf } \Omega_i
\end{align*}
beispielsweise $\otimes_{i\in I}\mathcal{B}$ auf $\Omega=\mathbb{R}^I$ (Funktionenraum).

\begin{definition*}[Zylinder,Pfeiler]
	$Z \subseteq \Omega^I$ heißt Zylinder, wenn $Z=\pi_j^{-1}(C)=C\times\Omega^{j^C}$ wobei $J\subseteq I$ eine endliche Teilindexmenge ist und $C$ die endlich dim. Basis des Zylinders ist;
	und Pfeiler, wenn $C$ ein Rechteck $\times_{i \in J}A_i$ ist.
\end{definition*}

Auf den Pfeilern lässt sich das n-dim. Produktmaß erklären. $P^n(C)=\pi_{j\in J}P(A_j)$

Im abzählbar unendlichen Fall ist die Vorgangsweise ähnlich, wie im endl. dim. Fall:

\begin{theorem}[7.PR]
	$X_i$ sei eine Folge von SGn auf $(\Omega_i,\mathcal{A}_i)$ mit gegebener gemeinsamer Verteilung
	\begin{align*}
		P_i(A) = P[X_j\in A_j, j\leq i, X_i \in A]
	\end{align*}
	für $A_j\in\mathcal{A}_j$ und $A\in\mathcal{A}_i$. D.h. die gemeinsame Verteilung von $(X_1,...,X_n)$ sind auf $(\times_i\Omega_i, \otimes_i\mathcal{A}_i)$ mit
	\begin{align*}
		P[(X_1,...,X_n)\in A_n, X_i \in \mathbb{R}, i>n] = P[(X_1,...,X_n) \in A_n]
	\end{align*}
	also für Zylinder $Z = \pi_{1,...,n}^{-1}(C_n)$, $C_n \in \otimes_i\mathcal{A}_i$ gilt $P[Z]=P_n(Cn)$.
	
	Wenn diese Wahrscheinlichkeitsverteilung verträglich sind, d.h. die Randverteilungen (bzw. alle Teilmengen endl.) eindeutig sind, lässt sich obiges Prinzip verallgemeinern.
\end{theorem}

\begin{theorem}[8.PR Satz von Kolmogoroff]
	$(\Omega_i,\mathcal{A}_i), i\in I$ sei eine Familie von Messräumen und $P_j$ sind endl. dim. Verteilungen auf $(\mathbb{R}^J, \mathcal{B}_{|J|}), J \subset I, J$ endlich. Diese Verteilungen seien verträglich ($P_j=P_k\pi_{k,j}^{-1}, J \subset K, K$ ebenfalls endlich).
	
	Dann existiert ein eindeutiges Maß $P$ auf $(\mathbb{R}^I, \mathcal{B}_I=\otimes_{i\in I}\mathcal{B})$ mit genau diesen Randverteilungen ($P(A)=P_J(\pi_j^{-1}(A)), A \in \otimes_{i\in J}\mathcal{A}_i$).
\end{theorem}

Das Produktmaß $\mu_1\otimes\mu_2$ auf $\mathcal{A}_1\otimes\mathcal{A}_2$ muss nicht vollständig sein, wenn $(\Omega_i,\mathcal{A}_i,\mu_i), i=1,2$ vollständige Maßräume sind.

\begin{example}
	$(\mathbb{R}^2,\mathcal{L}_2,\lambda_2)$ ist ein vollständiger Maßraum. $\lambda_2(\mathbb{R}\times\{1\})=0$ und für beliebiges $A\subseteq\mathbb{R} \lambda_2(A\times\{1\})=0$. Wenn $A\notin\mathcal{L}$, dann sollte aber trotzdem der Schnitt von $A\times\{1\}$, $(A\times\{1\})_1=A\in \mathcal{L}$ messbar sein, das ist ein Widerspruch. Es folgt also $\mathcal{L}\otimes\mathcal{L}\neq\mathcal{L}_2$.
\end{example}

\begin{theorem}[9.PR]
	$(\Omega_i,\mathcal{A}_i,\mu_i), i=1,2$ seinen sigma-endliche Maßräume. Wenn die messbaren Funktionen $f_i:(\Omega_i,\mathcal{A}:i)\rightarrow(\mathbb{R},\mathcal{B})$ entweder
	\begin{itemize}
		\item $f_i\geq 0$ oder
		\item $f_i$ ist integrierbar, $i=1,2$
	\end{itemize}
	ist, dann gilt
	\begin{align*}
		\int f_1 \cdot f_2 d\mu_1\otimes d\mu_2 = \int f_1 d\mu_1 \cdot \int f_2 d\mu_2
	\end{align*}
	Im 2. Fall ist $f_1\cdot f_2$ auf $(\Omega_1\times\Omega_2, \mathcal{A}_1\otimes\mathcal{A}_2,\mu_1\otimes\mu_2)$ integrierbar.
\end{theorem}

\begin{proof}
	Eigentlich klar, da $(f_1\cdot f_2)_{x_1} = f_1(x_1)\cdot f_2(.)$ $\mathcal{A}_2$ messbar ist und nach Fubini
	\begin{align*}
		\int f_1 \cdot f_2 d\mu_1\otimes d\mu_2 = \int_{\Omega_1}\int_{\Omega_2} f_1(x_1) f_2(x_2) d\mu_2(x_2) d\mu_1 = \int_{\Omega_1}\left(f_1(x_1) \int_{\Omega_2} f_2(x_2) d\mu_2(x_2) \right) = \int f_1 d\mu_1 \cdot \int f_2 d\mu_2 
	\end{align*}
	Genauso gilt $\int |f_1f_2| d\mu_1\otimes d\mu_2 = \int |f_1| d\mu_1 \int |f_2| d\mu_2$ und wenn die rechte Seite endlich ist, gilt $f_1\cdot f_2 \in \mathcal{L}_1$ bezüglich dem Produktraum.
\end{proof}

Die Betrachtung unabhängiger SGn $X_i, i=1,...,k$ erfolgt bequemer auf dem Produktraum. Auch wenn alle $\Omega_i=\Omega$, also alle SGn auf dem selben Raum definiert sind, übersiedelt man für die Erklärung der gemeinsamen Verteilung auf den Produktraum.

Wenn $X=(X_1,...,X_k)$ ein Vektor unabhängiger SGn $X_i$ ist, gilt
\begin{align*}
	PX^{-1} = PX_1^{-1}\otimes ...\otimes PX_k^{-1}.
\end{align*}
Wenn $PX^{-1}$ ein Wahrscheinlichkeitsmaß mit Dichte ist, dann gilt
\begin{align*}
	PX^{-1}(A) = \int_A f_X d\lambda_k = \int_A f(x_1,...,x_k) d\lambda(x_1)...d\lambda(x_k)
\end{align*}
und die Randverteilung von $X_i$ ist mit Dichte
\begin{align*}
	PX_i^{-1}(A_i) = \int_{A_i\times\mathbb{R}^{k-1}} f_X d\lambda_k = \int_{A_i}\left(\int_\mathbb{R}...\int_\mathbb{R} f_X d\lambda ... d\lambda \right) d\lambda
\end{align*}
und wieder erhält man die Randdichte
\begin{align*}
	f_i(x_i) = \int_\mathbb{R} ...\int_\mathbb{R} f(x_1,...,x_k) d\lambda(x_1) ... d\lambda(x_{i-1}) d\lambda(x_{i+1}) ... d\lambda(x_k)
\end{align*}
(Das ist nicht neu, aber jetzt wird kein Umweg über das Riemann-Integral benötigt.)

Wenn die $X_i$ unabhängig sind ist die gemeinsame Dichte
\begin{align*}
	f_X(x_1,...,x_k) = \prod_{i=1}^{k}f_i(x_i).
\end{align*}
Nach dem letzten Satz gilt für integrierbare $f$ und $g$ und unabhängige $X$ und $Y$
\begin{align*}
	\mathbb{E}f(X)g(Y) = \mathbb{E}f(X)\cdot\mathbb{E}g(Y)
\end{align*}
d.h. auch $\mathbb{E}XY = \mathbb{E}X \mathbb{E}Y$ und die SGn $X$ und $Y$ sind unkorreliert.

Die Umkehrung gilt natürlich nicht, da Unkorreliertheit nur bedeutet, dass es keinen linearen Zusammenhang gibt.

Wenn $Y=c$ f.s., dann sind $X$,$Y$ unabhängig, wenn $\mathbb{E}XY = \mathbb{E}X\mathbb{E}Y$, dass dann automatisch gilt. Auch wenn $X\sim A_{p_1}$ (Alternativ-verteilt) und $Y\sim A_{p_2}$ und
\begin{align*}
	\mathbb{E}XY = \mathbb{E}X\mathbb{E}Y = 0(1-p_1p_2) + 1p_1p_2 = p_1p_2
\end{align*}
d.h. $P(X=1)P(Y=1)=p_1p_2$ und $X$ und $Y$ sind unabhängig.

Ansonsten ist nur bei Normalverteilung kein Unterschied zwischen Unkorreliertheit und Unabhängigkeit.

\begin{example}
	$(X,Y)\sim \mathcal{N}(\mu_x,\mu_y,\sigma_x^2,\sigma_y^2,\rho)$ o.B.d.A $\mu_x=\mu_y = 0$
	
	Die gemainsame Dichte zerfällt (siehe EI24)
	\begin{align*}
		f(x,y) = \underbrace{\frac{1}{\sqrt{2\pi}\sigma_x} exp\left(-\frac{1}{2}\left(\frac{x}{\sigma_x}\right)^2\right)}_{f_1(x)} \cdot \underbrace{\frac{1}{\sqrt{2\pi}\sigma_x} \frac{\sigma_x}{\sigma_y\sqrt{1-\rho^2}} exp\left(-\frac{1}{2}\left(\frac{\sigma_x^2(y-m)^2}{(1-\rho^2)\sigma_y^2}\right)^2\right)}_{f_2(x,y)}
	\end{align*}
	mit $m=\rho x \frac{\sigma_y}{\sigma_x}$ d.h.
	\begin{align*}
		\mathbb{E}XY = \int\int xyf(x,y)d\lambda_2 = \int\underbrace{\int y f_2(x,y) dy}_m xf_1(x) dx = \int m x f_1(x) dx = \rho \int x^2 f_1(x) dx = \rho \frac{\sigma_y}{\sigma_x}\sigma_x^2 = \rho\sigma_x\sigma_y
	\end{align*}
	$X$,$Y$ sind unkorreliert $\iff \rho = 0$, dann ist $f(x,y) = f_1(x)\cdot \Phi(\frac{y}{\sigma_y})$ ($\Phi$ Dichte der $\mathcal{N}(0,1)$) und sind $X$,$Y$ unabhängig.
\end{example}

Zwei Normalverteilungen können nur linear abhängen. Ansonsten kann sogar eine vollständige Abhängigkeit (nicht linear) bei unkorrelierten SGn vorliegen.

\begin{example}
	$X\sim U_{-1,1}, Y = X^2$. Dann gilt $\mathbb{E}X=0$ und $\mathbb{E}XY = \mathbb{E}X^3 = 0$ und $X$ und $Y$ sind unkorreliert.
\end{example}

Nur wenn für alle integrierbaren $f$,$g$ $f(X)$ und $g(Y)$ unkorreliert sind, dann sind $X$ und $Y$ unabhängig.

Der Satz von Fubini ist ein wichtiges Werkzeug auch um bekannte Sätze der Integrationstheorie zu verallgemeinern, wie beispielsweise die partielle Integration.

\begin{theorem}[10.PR]
	$\mu_F$ und $\mu_G$ seinen Lebesgue-Stieltes Maße mit $F$ bzw. $G$ als Verteilungsfunktionen. $G_-(x)=\lim_{\tilde{x}\nearrow x}G(\tilde{x})$ ist der linksseitige Grenzwert von $G$. Dann gilt
	\begin{align*}
		\int_{(a,b]}Fd\mu_G + \int_{(a,b]}G_-d\mu_F = F(b)G(b)-F(a)G(a)
	\end{align*}
\end{theorem}
\begin{proof}
	in der Übung.
	
	Wenn $F$ und $G$ stetig differenzierbar sind, dann gilt $d\mu_G = G'd\lambda$ und $d\mu_F = F'd\lambda$ und
	\begin{align*}
		\int_{(a,b]} F\cdot G' d\lambda + \int_{(a,b]} G\cdot F' d\lambda = F(b)G(b)-F(a)G(a)
	\end{align*}
	und in der üblichen Schreibweise für Stammfunktionen $\int FG' = FG - \int F'G$
\end{proof}

Mit dem Hauptsatz der Diff- u. Integrationstheorie lässt sich auch die bedingte Verteilung auf stetige Verteilungen erweitern.

Für diskrete SGn $X$,$Y$ (beispielsweise auf $\mathbb{N}$ verteilt) ist
\begin{align*}
	P[X=k|Y=l] = \frac{P[X=k,Y=l]}{P[Y=l]}, k,l\in\mathbb{N}
\end{align*}
die Punktwahrscheinlichkeit $p_k$ der bedingten Verteilung $X|Y=l$.

Besitzt $X$ eine stetig differenzierbare VF $F$ mit $F'=f_X$ als Dichte, gilt
\begin{align*}
	\lim\limits_{\triangle\rightarrow0}\frac{F(x+\triangle)-F(x)}{\triangle} = f_X(x) \text{ oder}\\
	\frac{P[X\in[x-\triangle,x]]}{\triangle} \rightarrow f_X(x) \text{ für } \triangle\rightarrow0 \text{ oder}\\
	P[X\in[x-\triangle,x]] \sim f_X(x)\triangle
\end{align*}

Mit dieser "infidezimalen Wahrscheinlichkeit" als Dichte erhält man, wenn auch $Y$ die Dichte $f_Y$ hat,
\begin{align*}
	P[X\in[x-\triangle,x]|Y\in[y-\triangle,y]] = \frac{P[X\in[x-\triangle,x], Y\in[y-\triangle,y]]}{P[Y\in[y-\triangle,y]]} =\\
	\frac{\int_{x-\triangle}^{x}\int_{y-\triangle}^{y} f_{X,Y}(s,t) ds dt}{\int_{[y-\triangle,y]}f_Y(t) d\lambda(t)}\sim \frac{f_{X,Y}(x,y)\triangle^2}{f_Y(y)\triangle} = \frac{f_{X,Y}(x,y)\triangle}{f_Y(y)}
\end{align*}

\begin{definition}[bedingte Dichte]
	$X:\Omega\rightarrow\mathbb{R}$ und $Y:\Omega\rightarrow\mathbb{R}$ sind SGn mit Dichte $f(x,y)$. $f_X$, $f_Y$ sind die Randdichten von $X$ und $Y$. Dann heißt für $y$ mit $f_Y(y)>0$
	\begin{align*}
		f(x|y) := \frac{f(x,y)}{f_Y(y)}
	\end{align*}
	die bedingte Dichte von $X$ bedingt durch $Y=y$. $f(x|y)$ ist $PY^{-1}_-$ f.s. definiert.
\end{definition}

Da für festes $y$
\begin{align*}
	\int_\mathbb{R} f(x|y) d\lambda(x) = \frac{\int_\mathbb{R} f(x,y) d\lambda(x)}{\int_\mathbb{R} f(x,y) d\lambda(x)} = 1 
\end{align*}
und $f(x|y)\geq 0$ ist $f(.|y)$ eine Wahrscheinlichkeitsdichte und tatsächlich eine Verteilung festgelegt.

Wenn $X|Y=y PY^{-1}$-f.s. einen endlichen Erwartungswert besitzt, dann heißt die Funktion
\begin{align*}
	y\mapsto \mathbb{E}[X|Y=y] = \int x f(x|y) d\lambda(x)
\end{align*} 
bedingter Erwartungswert. Dann ist $h(Y):=\mathbb{E}[X|Y]$ eine $PY^{-1}$-f.s. messbare Funktion. $h(.)$ heißt auch Regressionsfunktion. $h(Y)$ ist als Prognose von $X$ nach der Beobachtung von $Y$ zu verstehen.

\begin{example}
	$(X,Y)$ sei bivariat normalverteilt $(X,Y)\sim\mathcal{N}(\mu_x,\mu_y,\sigma_x^2,\sigma_y^2, \rho)$. Die Darstellung der bivariaten Normalverteilungsdichte (PR18) führt auf die bedingte Dichte
	\begin{align*}
		f(x|y) = \frac{1}{\sqrt{2\pi v}}exp\left(-\frac{1}{2}\left(\frac{x-m}{v}\right)^2\right)\\
		\text{mit } m = \mu_x + \rho\frac{\sigma_x}{\sigma_y}(y-\mu_y) \text{ und } v^2 = \sigma_x^2(1-\rho^2)
	\end{align*}
	d.h. $X|Y\sim\mathcal{N}(m,v^2)$.
	
	Die Regressionsfunktion ist linear
	\begin{align*}
		\mathbb{E}[X|Y] = \mu_x + \rho \frac{\sigma_x}{\sigma_y}(y-\mu_y)
	\end{align*}
	Auch diese Eigenschaft charakterisiert die Normalverteilung.
\end{example}

Für die Regressionsfunktion $H(Y):=\mathbb{E}[X|Y]$ gilt bei Normalverteilung $\mathbb{E}H(Y)=\mu_x$. Das ist generell der Fall:
\begin{align*}
	\mathbb{E}H(Y) = \int\mathbb{E}[X|Y=y]dPY^{-1} = \int\int x f(x|y) dx f_Y(y) dy\\
	= \int\int x \underbrace{f(x|y)f_Y(y)}_{f(x,y)} dy dx =	\int\int x f(x,y) dy dx = \int x f_X(x) dx = \mathbb{E}X.
\end{align*}
Die mittlere Prognose entspricht dem unbedingten Erwartungswert.

\subsection{Faltung von Maßen}
\begin{definition}[Faltungsmaß]
	$\mu_1,\mu_2$ sind sigma-endliche Maße auf $(\mathbb{R}, \mathcal{B})$. Das Faltungsmaß ist durch
	\begin{align*}
		\mu_1*\mu_2 (A) = \int \mu_1(A-y) \mu_2(dy)
	\end{align*}
	definiert.
	
	Das $\mu_1 * \mu_2$ tatsächlich ein Maß ist, ergibt sich daraus, dass es das von der Summe $S=X_1+X_2$ erzeugte Maß ist. D.h. $\mu_1 * \mu_2 = (\mu_1 \otimes \mu_2)S^{-1}$
\end{definition}

\begin{proof}
	$S^{-1}(A) = \{(x,y)\in \mathbb{R}^2: x+y\in A\}$, wobei $A \in \mathcal{B}$. Der Schnitt von $S^{-1}(A)$ ist $S^{-1}(A)_y = \{x|x\in A-y\} = A-y$
	
	Die Darstellung des Produktmaßes mittels Schnitten wie zuvor,
	\begin{align*}
		\mu_1\otimes\mu_2(B) = \int \mu_1(B_y) d\mu_2(y), B \in \mathcal{A}_1 \otimes \mathcal{A}_2
	\end{align*}
	ergibt
	\begin{align*}
		\mu_1 * \mu_2(S^{-1}(A)) = \int \mu_1(A-y) d\mu_2(y) \text{ und }\\
		(\mu_1\otimes\mu_2)S^{-1} = \mu_1 * \mu_2.
	\end{align*}
	Es ist auch in die andere "Richtung" darstellbar: $\mu_1 * \mu_2 = \int \mu_2(A-x) d\mu_1(dx)$
\end{proof}

Neben der Kommutativität $\mu_1 * \mu_2 = \mu_2 * \mu_1$ besitzt die Faltung noch folgende Eigenschaften:
\begin{itemize}
	\item $\mu_i, i=1,2,3$ sind sigma-endliche Maße auf $(\mathbb{R}, \mathcal{B})$
	
	\begin{align*}
		(\mu_1*\mu_2)* \mu_3 = \mu_1 * (\mu_2 * \mu_3)
	\end{align*}
	\begin{proof}
		$A \in \mathcal{B}_2$
		\begin{align*}
			(\mu_1 * \mu_2) * \mu_3(A) = \int \mu_1 * \mu_2 (A-z) d\mu_3(z) = \int\left(\int \mu_2(A-z-x) d\mu_1(x)\right) d\mu_3 = \\
			\int\int \mu_2(A-z-x) d\mu_3(z) d\mu_1 = \int \mu_2 * \mu_3 (A-x) d\mu_1(x) = (\mu_2 * \mu_3) * \mu_1(A) = \mu_1 * (\mu_2 * \mu_3) (A)
		\end{align*}
	\end{proof}

	\item $\mu_1(\mathbb{R}) = \mu_2(\mathbb{R}) = 1 \implies \mu_1 * \mu_2(\mathbb{R}) = 1$
	
	\begin{proof}
		Da $\mathbb{R}-y=\mathbb{R} \forall y\in\mathbb{R}$ gilt
		\begin{align*}
			\mu_1 * \mu_2 (\mathbb{R}) = \int \mu_2(\mathbb{R}-x) d\mu_1(x) = \int \mu_2(\mathbb{R}) d\mu_1(x) = \mu_2(\mathbb{R}) \cdot \mu_1(\mathbb{R}) = 1
		\end{align*}
		
		Die Faltung von Wahrscheinlichkeitsmaßen ist auch ein Wahrscheinlichkeitsmaß.
	\end{proof}
	
	Sind $\mu_1 = PX_1^{-1}$ und $\mu_2=PX_2^{-1}$ von SGn $X_1$ bzw. $X_2$ induziert, dann ist $\mu_1 * \mu_2 = PX_1^{-1} * PX_2^{-1} = P(X_1+X_2)^{-1}$ das von $X_1,X_2$ induzierte Maß, vorausgesetzt $X_1,X_2$ sind unabhängig, also für $X=(X_1,X_2), PX = PX_1^{-1} \otimes PX_2^{-1}$.
	
	\item Es existiert auch ein neutrales Element der Faltung $\mu_0$ mit $\mu_0 * \mu = \mu = \mu * \mu_0$.
	
	\begin{proof}
		Für $\mu_0 = \delta_0$, d.h. $\delta_0(A) = 1_A(0)$ gilt
		\begin{align*}
			\mu_0 * \mu(A) = \int \mu_0(A-x) d\mu(x) = \int \underbrace{1_{A-x}(0)}_{1_A(x)} d\mu(x) = \int 1_A(x) d\mu(x) = \mu(A).
		\end{align*}
	\end{proof}
\end{itemize}

Die Faltung bildet eine kommutative Halbgruppe.

Die bereits vorher erklärte Faltung von messbaren Funktionen hängt erwartungsgemäß mit der Faltung von Maßen zusammen. Bei Maßen mit Dichten ist die Dichte der Faltung genau die Faltung der Dichten.

\begin{theorem}[11.PR]
	$\mu_1,\mu_2$ seien Maße mit Dichten bezüglich $\lambda$: $\mu_1 = \int f_1 d\lambda$ und $\mu_2 = \int f_2d\lambda$ und $f_1,f_2$ sind reellwertig.
	
	Dann ist
	\begin{align*}
		\mu_1 * \mu_2(A) = \int_A\left(\int_\mathbb{R} f_1(s-y) f_2(y) d\lambda(y) \right) d\lambda(s)
	\end{align*}
	 also die Dichte von $\mu_1 * \mu_2$ ist die Faltung der Dichten $f_1 * f_2$.
\end{theorem}
\begin{proof}
	\begin{align*}
		\mu_1 * \mu_2 (A) = \int_{\mathbb{R}} \mu_1(A-y) d\mu_2(y) = \int_{\mathbb{R}} \int_{A-y} f_1(x) d\lambda(x) f_2(y) d\lambda(y)
	\end{align*}
	Die Transformation $T_y(x) = x-y$ ist als lineare Transformation für $\lambda$ translationsinvariant. Für jedes $y$ gilt
	\begin{align*}
		\lambda T_y^{-1}(.) = \lambda(T_y^{-1}(.)) = \lambda(.)
	\end{align*}
	Das innere Integral ist nach dem Transformationssatz
	\begin{align*}
		\int_{A-y} f_1(x) d\lambda = \int_{T_y(A)} f_1 d\lambda = \int_A f_1\circ T_y(s) d\lambda(s)
	\end{align*}
	und mit dem Satz von Fubini ergibt sich
	\begin{align*}
		\mu_1 * \mu_2 (A) = \int_{\mathbb{R}} \left(\int_A f_1(s-y) d\lambda(s) \right) f_2(y) d\lambda(y) = \int_A \underbrace{\int_{\mathbb{R}} f_1(s-y) f_2(y) d\lambda(y)}_{f_1 * f_2} d\lambda(s)
	\end{align*}
	Analog gilt natürlich auch
	\begin{align*}
		\mu_1 * \mu_2 (A) = \int_A \left(\int f_2(s-x) f_1(x) d\lambda(x) \right) d\lambda(s).
	\end{align*}
\end{proof}

Damit wurde die Faltung im stetigen Fall auf die Faltungsdichte zurückgeführt. Analog werden diskrete Maße gefaltet.

\begin{theorem}[12.PR]
	$\mu_i, i=1,2$ sind diskrete Maße auf $(\mathbb{R}, \mathcal{B})$ mit Trägermenge $D_i$ (abzählbar). Dann ist $\mu_1 * \mu_2$ diskret verteilt mit der Trägermenge $D = D_1 + D_2 = \{s|s=x+y, x\in D_1, y\in D_2\}$ und für $s\in \mathbb{R}$ gilt
	\begin{align*}
		\mu_1 * \mu_2(s) = \sum_{y \in D_2} \mu_1(\{s-y\})\mu_2(\{y\})
	\end{align*}
\end{theorem}
\begin{proof}
	\begin{align*}
		\mu_1 * \mu_2 (A) = \int_{D_2} \mu_1(A-y) d\mu_2(y) = \sum_{y\in D_2} \mu_1(A-y)\mu_2(\{y\})
	\end{align*}
	Die Trägermenge von $\mu_1 * \mu_2$ ist $D$, da $D^c-y \subseteq D_1^c$ für $y\in D_2$ und $\mu_1(D_1^c)=0$ gilt
	\begin{align*}
		\mu_1 * \mu_2 (D^c) = \sum_{y \in D_2} \underbrace{\mu_1(D^c-y)}_{=0}\mu_2(\{y\})\\
		\implies \mu_1 * \mu_2 (D^c) = 0
	\end{align*}
	$\mu_1 * \mu_2(A) = \mu_1 * \mu_2 (A\cap D)$ und für $s\in D$ gilt
	\begin{align*}
		\mu_1 * \mu_2 (\{s\}) = \sum_{y \in D_2} \mu_1(\{s-y\})\mu_2(\{y\})
	\end{align*}
	oder mit vertauschten $\mu_i, i=1,2$
	\begin{align*}
		\mu_1 * \mu_2 (\{s\}) = \sum_{x \in D_1} \mu_2(\{s-x\}) \mu_1(\{x\})
	\end{align*}
\end{proof}

Für stochastischen Größen und die entsprechenden induzierten Wahrscheinlichkeitsmaße wurde diese diskrete Faltung bereits erklärt, o.B.d.A sei $X$ auf $\mathbb{N}$ diskret verteilt, genauso wie $Y$, dann ist
\begin{align*}
	P(X+Y=k) = \sum_{m\geq 0} P(X=k-m) P(Y=m)
\end{align*}
wenn $X$,$Y$ unabhängig sind, bzw.
\begin{align*}
	P(X+Y=k) = \sum_{m\geq 0} P(X=k-m|Y=m) P(Y=m)
\end{align*}
wenn $X$,$Y$ nicht unabhängig sind.

Mit der bedingten Dichte gilt auch im stetigen Fall für die Dichte von $X+Y=:S$ bei abhängigen $X$ und $Y$
\begin{align*}
	f_S(s) = \int f(s-t|Y=t) f_Y(t) d\lambda(t)
\end{align*}
wobei $f(x|y)$ die Dichte der bedingten Verteilung von $X|Y$ bezeichnet. Auch hier können die marginalen Dichten getauscht werden, d.h. mit $g(y|x)$ als bedingte Dichte $Y|X$
\begin{align*}
	f_S(s) = \int g(s-x|X=x) f_X(x) d\lambda(x).
\end{align*}

Für die Binomialverteilung wurde die Faltung bereits durchgeführt.

\begin{example}[Faltung negativer Binomialverteilung]
	$X$ ist $NegB_{k,p}$, wenn $X$ die Anzahl der Fehlversuche ($\backsimeq 0$) bis zum k-ten Erfolg ($\backsimeq 1$) bei unabhängigen $0-1$ Versuchen mit Erfolgswahrscheinlichkeit $p$.
	\begin{align*}
		P(X=m)=\binom{m+k-1}{k-1}p^k(1-p)^m
	\end{align*}
	Die geometrische Verteilung $G_p$ ist ein Spezialfall, $G_p=NegB_{1,p}$ (Variante 2 der geometrischen Verteilung ist hier gemeint).
	\begin{align*}
		X_i\sim G_p: P(X_i=m)=p(1-p)^m, m\in\mathbb{N}
	\end{align*}

	Die Faltung zweier $G_p$ heißt
	\begin{align*}
		P(X_1+X_2=m)=\sum_{l=0}^{m}p(1-p)^{m-l}p(1-p)^l = (m+1)p^2(1-p)^m = \binom{m+2-1}{2-1}p^2(1-p)^m
	\end{align*}
	daher $X_1+X_2\sim NegB_{2,p}$.
	
	Genauso ergibt die Faltung $NegB_{k,p}*G_p=:\mu$
	\begin{align*}
		\mu(\{m\}) = \sum_{i=0}^{m}\binom{i+k-1}{k-1}p^k(1-p)^ip(1-p)^{m-i} = p^{k+1}(1-p)^m\sum_{i=0}^{m}\binom{i+k-1}{k-1} = p^{k+1}(1-p)^m\binom{k+m}{k}
	\end{align*}
	Mit dem Pascalschen Dreieck $\binom{n+1}{k} = \binom{n}{k-1}\binom{n}{k}$ und Induktion kann gezeigt werden, dass $\sum_{i=0}^{m}\binom{i+k-1}{k-1} = \binom{k+m}{k}$.
	
	Somit ist $NegB_{k,p} * G_p = NegB_{k+1,p}$. Mit Induktion gilt daher sofort $NegB_{k,p}*NegB_{\tilde{k},p} = NegB_{k+\tilde{k},p}$.
	
	Die Summe unabhängiger $NegB$-Verteilungen (mit gleichem $p$) ist wieder negativ Binomialverteilt.
	
	Bei verschiedenen $p$ ist die Summe nicht nach einer $NegB$ verteilt.
\end{example}

Die Faltung eines (absolut) stetigen Maßes und eines diskreten Maßes ergibt wieder ein Maß mit Dichte.

\begin{theorem}[13.PR]
	$\mu$ sei ein Maß mit Dichte $f$ bez. $\lambda$. $\mu$ daher $\mu \ll\lambda$ und $\nu$ sei ein diskretes Maß mit Trägermenge $D$ (abzählbar) auf $(\mathbb{R},\mathcal{B})$. Dann ist $\mu * \nu \ll \lambda$ und
	\begin{align*}
		\mu * \nu (A) = \int_A \underbrace{\sum_{k\in D} f(x-k)\nu(\{k\})}_{\text{ Dichte von }\mu * \nu} d\lambda(x).
	\end{align*}
\end{theorem}
\begin{proof}
	Da $\mu \ll \lambda$ gilt für eine Nullmenge $N$, $\lambda(N) = 0$ auch $\mu(N) = 0$. Wegen der Translationsinvarianz $\lambda(N-s) = 0$ für alle $s \in \mathbb{R}$, daher auch
	\begin{align*}
		\mu(N-s)=0 \text{ und } \mu * \nu (N) = \int \mu(N-s) \nu(ds) = 0\\
		\implies \mu * \nu \ll \lambda.
	\end{align*}

	Sei $A = (-\infty, t], t\in \mathbb{R}$, dann gilt
	\begin{align*}
		\mu * \nu (A) = \int \sum_{k \leq t-s, k\in D} \nu(\{k\}) f(s) d\lambda(s) = \int \sum_{k\in D} \nu(\{k\}) 1_{(-\infty, t]}(k+s) f(s) d\lambda(s) = \\
		\sum_{k\in D} \nu(k) \int 1_{(-\infty,t]}(x) f(x-k) d\lambda(x) = \int_{(-\infty,t]} \sum_{k\in D} f(x-k) \nu(k) d\lambda(x)
	\end{align*}
	Die Aussage gilt daher für Intervalle. Wegen des Fortsetzungssatzes gilt sie auch für jedes $A \in \mathcal{B}$.
\end{proof}
\begin{remark}
	Bei mehr als 2 Maßen genügt ein Maß $\mu_i \ll \lambda$.
\end{remark}
\begin{example}
	$\mu$ sei eine Exponentialverteilung $E_{x_1}$ mit Dichte $f(s)=e^{-s}1_{(0,\inf)}(s)$. $\nu$ entspreche einer Geometrischen Verteilung $G_{\frac{1}{2}}$ (Version 1). $\nu(\{k\}) = \frac{1}{2^k}, k\geq 1$.
	
	Die Lebesgue-Dichte des Faltungsmaßes ist
	\begin{align*}
		h(x) = \sum_{k\geq 1}\frac{1}{2^k} e^{-(x-k)} 1_{(0,\infty)}(x-k) = e^{-(x-1)}\frac{(\frac{e}{2})^{\lfloor x\rfloor} - 1}{e-2}
	\end{align*}
	was einer stückweisen (auf $[n,n+1)$) gewichteten Exponentialfunktion (Verteiliung) entspricht.
\end{example}

Die Faltung kann auch mit der Verteilungsfunktion durchgeführt werden:

$X$ besitzt $F$ als Verteilungsfunktion $Y$ hat $G$ als VF, $X$ und $Y$ unabhängig. Die Verteilungsfunktion von $X+Y$ ist
\begin{align*}
	H(t) = \int F(t-s) dG(s) = \int G(t-s) dF(s)
\end{align*}
(siehe Übung).

Wie im letzten Beispiel ist $H(.)$ bis auf eine Nullmenge differenzierbar, wenn $F$ und $G$ differenzierbar mit Dichten $F'=f$ und $G'=g$ sind. Die (stückweise) differenzierbar VF $H$ ergibt die Dichte aus Satz 11.PR.
\begin{align*}
	\frac{dH}{dx}(s) = \int \frac{dF}{dx}(s-t)\frac{dG}{dx}(t) d\lambda(t).
\end{align*}

Natürlich wird auch die Verteilung der Differenz $X-Y$ unabhängiger Sgn $X,Y$ mit der Faltung von $X$ und $-Y$ bestimmt.

Wenn Dichten existieren ist die Dichte von $X-Y$ $h(t)=\inf f_X(s) f_Y(s-t) d\lambda(s)$. Die Wahrscheinlichkeit für $X \leq Y$ kann auch mit der FaltungsVF für $X-Y$ berechnet werden.

\begin{lemma}
	$X,Y$ SGn mit VF $F_X,F_Y$ und Dichten $f_X,f_Y$. Wenn $X$ u.a. $Y$, dann ist
	\begin{align*}
		P[X\leq Y] = \int_\mathbb{R} F_X(t)f_Y(t) d\lambda(t).
	\end{align*}
\end{lemma}
\begin{proof}
	\begin{align*}
		P[X\leq Y] = \int_{\{x\leq y\}} f(x,y) d\lambda_2 = \int_\mathbb{R} \int_{(-\infty,Y]} f_X(x) f_Y(y) d\lambda_2(x,y) = \int_\mathbb{R} F_X(y)f_Y(y) d\lambda(y)
	\end{align*}
\end{proof}
\begin{example}
	$X,Y$ unabhängige Exponentialverteilungen mit Raten $\lambda_1,\lambda_2$, d.h. $X\sim E_{x_{\lambda_1}}, Y \sim E_{x_{\lambda_2}}$.
	\begin{align*}
		P(X\leq Y) = \int_{\mathbb{R}^+} (1-e^{-\lambda_1y})\lambda_2 e^{-\lambda_2y}d\lambda(y) = 1 - \frac{\lambda_2}{\lambda_1+\lambda_2} = \frac{\lambda_1}{\lambda_1 + \lambda_2}
	\end{align*}
	Anteil der Raten von $X$.
\end{example}

\end{document}
